\documentclass[a4paper,titlepage,12pt]{article}
\usepackage[utf8]{inputenc}
\usepackage[T1]{fontenc}
\usepackage[margin=2cm]{geometry}
\usepackage{parskip}
\usepackage{graphicx}
\usepackage{hyperref}
\usepackage{listings}
\usepackage{multirow}
\usepackage{gensymb}
\usepackage[usenames,dvipsnames]{color}
\hypersetup{
	colorlinks,
	pdfauthor=Delan Azabani,
	pdftitle=Computer Graphics 200: OpenGL assignment
}
\lstset{basicstyle=\ttfamily, basewidth=0.5em}

\title{Computer Graphics 200:\\OpenGL assignment}
\date{November 1, 2014}
\author{Delan Azabani}

\pagenumbering{gobble}
\thispdfpagelabel0

\begin{document}

\maketitle
\pagenumbering{arabic}

\section{Design choices}

Only features from OpenGL 1.4 and its fixed function pipeline were used to
write this assignment. Although this API has long been deprecated, I chose it
for a handful of reasons:

\begin{itemize}
	\item OpenGL 1.4 is the only version supported by the lab machines,
	      Xubuntu \textit{and} Cygwin
	\item I found the syntax and semantics easier to learn than using
	      shaders and buffer objects
	\item There is a large set of established tutorials and documentation
	      for OpenGL 1.4
\end{itemize}

While the assignment lacks models relevant to the `Australian outback' theme
due to time constraints, every technical and interactive feature required by
the specification has been met, and even exceeded in some areas.

\section{System overview}

The common namespace prefix \texttt{cg200\_} will hereby be referred to as
\texttt{@} for brevity.

The assignment's code ultimately starts at \texttt{@main}, which calls
\texttt{@init} to configure a new window using GLUT. This window shall be
double buffered for smooth rendering, support an alpha channel for transparency
and fog, and contain a depth buffer to facilitate drawing of objects in a
temporal order which is nearly unconstrained. The function also binds callbacks
for events such as main loop iterations, key presses and window resizes.

\texttt{@start} turns on a large swath of features which are disabled by
default for performance:

\begin{itemize}
	\item \texttt{GL\_COLOR\_MATERIAL} allows objects to retain their
	      coloured surfaces after illumination
	\item \texttt{GL\_LIGHTING} allows this illumination instead of a
	      simple, global, invariant light
	\item \texttt{GL\_FOG} provides a simple global fog effect, which we
	      configure to be exponential
	\item \texttt{GL\_NORMALIZE} contributes to prevent lights from dimming
	      when zooming in
	\item \texttt{GL\_TEXTURE\_2D} enables texturing which is used for the
	      floor quadrilateral
	\item \texttt{GL\_DEPTH\_TEST} enables the use of the depth buffer to
	      correctly draw overlapping objects
	\item \texttt{GL\_BLEND} allows overlapping translucent fragments to
	      blend with one another
\end{itemize}

Finally the GLUT main loop is started, and after an initial \texttt{@reshape}
to configure the viewport, the process of rendering each frame occurs with a
repeated sequence of calls from \texttt{@idle} to \texttt{glutPostRedisplay} to
\texttt{@render}, which calls \texttt{@state\_init} \textit{once} to establish
global state.

Every frame starts with \texttt{@state\_tick}, which maintains accounting for
frame counts, time in the animation and process `worlds' and global scene
rotation along three axes. The frame is cleared to a gaudy bright magenta to
catch any accidental errors, then the camera is configured with a perspective
projection, location and direction using \texttt{@gluLookAt} and zoom is
applied.

\texttt{@render\_body} is then called after the scene rotations, before
swapping the frame buffers.

\newpage

\section{Objects and rendering}

A simple night sky is cleared by \texttt{@render\_sky}, then
\texttt{@render\_lights} creates the four lights positioned at the corners of
the scene, while also establishing global fog. This fog intensifies as the
scene becomes more distant by being inversely proportional to the zoom factor.

The floor of the scene is drawn with \texttt{@render\_floor} and consists of a
simple chequerboard texture which is dynamically generated and loaded once
while rendering the first frame, then subsequently used repeatedly. A sequence
of calls to \texttt{glMaterialfv} define the surface finishing for the floor,
which reflects no ambient light, all diffuse light, and all specular light,
while being slightly shiny with a small specular exponent.

The floor itself is not one quadrilateral, but 4096 smaller squares arranged
such that lighting calculations shall be more accurate. This is because OpenGL
calculates lighting for each vertex, and having more squares implies the
creation of more vertices.

The oscillating prism is simply a collection of four triangles drawn in a
single \texttt{GL\_TRIANGLES} sequence. It takes advantage of smooth shading to
yield faces that vaguely resemble colour gamut diagrams. The vertices that make
up each triangle have been carefully ordered, not only to maintain a consistent
anti-clockwise ordering for correct normals, but also to gracefully degrade
with four distinct coloured faces when flat shading is enabled. This is
achieved by having distinct colours for the \textit{final} vertex in each
triangle.

Motion for the prism is a combination of sinusoidal oscillation along the
positive half of the z-axis, along with monotonic varying rotations around all
three axial directions. Each axis has a rotation taking the form

$$120\degree\times\left(t+\sin\left(t+\frac{2k\pi}3\right)\right)$$

where $t$ is \texttt{STATE.time\_animation} and $k$ is an integer between zero
and two inclusive, arbitrarily assigned to the axis of rotation, and intended
to misalign the times when the rotations about each axis slow down. Because the
derivative of $t+\sin(t)$ is $1+\cos(t)$ and varies between zero and two
inclusive, the function is monotonic and the rotation of the prism shall never
reverse.

Transparency is satisfied by \texttt{@render\_square}, which merely draws an
elevated quadrilateral with translucent black. The aforementioned prism
repeatedly cuts through this quadrilateral, demonstrating the ability of
translucent objects to affect the surrounding colours.

\texttt{@render\_balls} is the most interesting function in the rendering
sequence. Using a sinusoidal, radial oscillation on the x-y-plane together with
a linear rotation around the x-axis allows the CMYK coloured balls to meet one
another in pairs. Level of detail in the sphere models is achieved with the
supporting \texttt{@calculate\_true\_sphere\_quality} function, which
quadratically increases the number of faces as the user zooms closer to the
scene.

To obtain full marks, key bindings and statistics are printed inside the OpenGL
window instead of \texttt{stdout}. A primitive form of in-band signalling is
used to allow rasterised text to consist of multiple fonts. Whenever
\texttt{@draw\_text} encounters a sequence matching
\texttt{{\textbackslash}a{\textbackslash}d+;} where
\texttt{{\textbackslash}a} is the ASCII BEL, the font is changed, with the
integer before the semicolon indexing \texttt{@fonts}.

\newpage

\section{Exceeding the specification}

The following interactive features surpass what is required by the assignment
specification:

\begin{itemize}
	\item \texttt{[uU]} toggles clockwise scene rotation around the z-axis,
	\item \texttt{[lL]} increases/decreases the base level of detail for
	      the spheres,
	\item \texttt{[+-]} increases/decreases the field of view from the
	      default $90\degree$ angle,
	\item \texttt{[rR]} resets all global state, including the entire
	      scene, and
	\item \texttt{[{\textbackslash}n]} switches the assignment to
	      fullscreen mode.
\end{itemize}

\section{Interesting facts}

\begin{itemize}
	\item While depth buffering \textit{mostly} frees developers from
	      having to sort their objects prior to rendering, some
	      restrictions still apply. Lights must be established before other
	      objects that shall be affected by them, and opaque objects must
	      be drawn before translucent objects that they may intersect when
	      projected.
	\item Two techniques must be combined to ensure that lights do not
	      erroneously change intensity while zooming with \texttt{glScalef}
	      --- \texttt{GL\_NORMALIZE} must be enabled, as well as a
	      \texttt{GL\_LINEAR\_ATTENUATION} which is inversely proportional
	      to the zoom factor.
	\item There doesn't appear to be a stack for material parameters, so
	      care must be taken to restore the surface finish values to their
	      defaults to avoid clobbering the rendering of other objects.
	\item To attain correct colours for rasterised text, lighting and fog
	      must be disabled prior to calling \texttt{glutBitmapCharacter}
	      and subsequently enabled afterwards.
	\item Depth testing must also be disabled temporarily while drawing
	      text, or some objects may obscure parts of the text in extreme
	      cases of model locations and zoom.
	\item Logical XOR can easily be achieved with \texttt{!=} even though
	      C doesn't have a \texttt{\^{}\^{}} operator.
\end{itemize}

\newpage

\section{References}

\begin{itemize}
\item
Clark, R. (2012)
\textit{OpenGL:Tutorial Framework:Light and Fog}.
Retrieved from \\
http://content.gpwiki.org/index.php/OpenGL:Tutorials:Tutorial\_Framework
\item
Eusebeîa. (2014)
\textit{The Tetrahedron}. \\
Retrieved from
http://eusebeia.dyndns.org/4d/tetrahedron
\item
Gamedev.net. (2012)
\textit{NeHe Productions - Everything OpenGL}. \\
Retrieved from
http://nehe.gamedev.net/
\item
Hennig, M. (2011)
\textit{opengl --- Setting glutBitmapCharacter color? --- Stack Overflow}. \\
Retrieved from
http://stackoverflow.com/a/8239654
\item
Khronos Group. (2002)
\textit{The relationship between glScalef() and light intensity}. \\
Retrieved from
https://www.opengl.org/discussion\_boards/showthread.php/154816
\item
Khronos Group. (2001)
\textit{Transparency, Translucency, and Blending}.
Retrieved from \\
https://www.opengl.org/archives/resources/faq/technical/transparency.htm
\item
Kilgard, M. J. (2000)
\textit{Avoiding 16 Common OpenGL Pitfalls}.
Retrieved from \\
http://www.opengl.org/archives/resources/features/KilgardTechniques/oglpitfall/
\item
Kilgard, M. J. (1996)
\textit{glutSolidCube, glutWireCube}.
Retrieved from \\
https://www.opengl.org/documentation/specs/glut/spec3/node82.html
\item
Kilgard, M. J. (1996)
\textit{glutSolidSphere, glutWireSphere}.
Retrieved from \\
https://www.opengl.org/resources/libraries/glut/spec3/node81.html
\item
Microsoft Corporation. (2012)
\textit{glMaterialfv function}.
Retrieved from \\
http://msdn.microsoft.com/en-us/library/windows/desktop/dd373945.aspx
\item
Silicon Graphics, Inc. (2006)
\textit{glColorMaterial}.
Retrieved from \\
https://www.opengl.org/sdk/docs/man2/xhtml/glColorMaterial.xml
\item
Silicon Graphics, Inc. (2006)
\textit{glLight}.
Retrieved from \\
https://www.opengl.org/sdk/docs/man2/xhtml/glLight.xml
\item
Silicon Graphics, Inc. (2006)
\textit{glRotate}.
Retrieved from \\
https://www.opengl.org/sdk/docs/man2/xhtml/glRotate.xml
\item
Silicon Graphics, Inc. (2006)
\textit{glShadeModel}.
Retrieved from \\
https://www.opengl.org/sdk/docs/man2/xhtml/glShadeModel.xml
\item
Urquhart, D. (2010)
\textit{OpenGL 2 Tutorials --- Swiftless Tutorials}. \\
Retrieved from
http://www.swiftless.com/opengltuts.html
\end{itemize}

\end{document}
